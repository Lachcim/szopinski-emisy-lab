\documentclass{article}

\usepackage{graphicx}
\usepackage{pdfpages}
\usepackage{hyperref}
\usepackage{textgreek}
\setlength{\parskip}{1em}

\begin{document}

	\title{EMISY lab 1 report:\\
	Interfacing the 8051 MCU with an LCD}
	\author{Michał Szopiński 300182\\\\
	https://github.com/Lachcim/szopinski-emisy-lab}
	\date{April 21, 2021}
	\maketitle
	
	\setcounter{section}{-1}
	\section{Abstract}
	
	The goal of this laboratory was to connect the AT89C4051 (Intel 8051
	family) microcontroller to an HD44780-like alphanumeric LCD screen. This
	involves designing the interfacing hardware and firmware.
	
	\section{Hardware implementation}
	
	Below is a schematic of the designed circuit.
	
	The microcontroller is connected to the power supply through a 100 nF
	decoupling capacitor to protect it from noise. The reset pin is pulled low
	through a 10 k\textOmega {} resistor. The resistor, together with a 10
	\textmu F electrolytic capacitor, forms a power-on reset circuit. Clock
	signal is provided by a 12 MHz crystal oscillator. This results in a
	processing speed of 1 machine cycle per microsecond.
	
	The LCD is also connected to the power rail through a decoupling capacitor.
	This is to protect its internal logic circuit from noise. The anode of the
	backlight LED array is connected through a 5.6 \textOmega {} resistor,
	providing a current of about 140 mA as prescribed in the datasheet.
	A potentiometer-based voltage divider is connected to the $V_o$ pin,
	allowing the user to adjust the contrast.
	
	Finally, the 8-bit data bus of the display is connected to port 1 of the
	microcontroller. When in 4-bit mode, only 4 pins of the bus are utilized.
	The register select (RS) and chip enable pin (E) are connected to pins 1
	and 0 of the third port.
	
	\includepdf[landscape]{schematic}
	
\end{document}
