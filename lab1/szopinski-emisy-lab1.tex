\documentclass{article}

\usepackage{graphicx}
\usepackage{pdfpages}
\usepackage{hyperref}
\usepackage{textgreek}
\setlength{\parskip}{1em}

\begin{document}

	\title{EMISY lab 1 report:\\
	Interfacing the 8051 MCU with an LCD}
	\author{Michał Szopiński 300182\\\\
	https://github.com/Lachcim/szopinski-emisy-lab}
	\date{April 21, 2021}
	\maketitle
	
	\setcounter{section}{-1}
	\section{Abstract}
	
	The goal of this laboratory was to connect the AT89C4051 (Intel 8051
	family) microcontroller to an HD44780-like alphanumeric LCD screen. This
	involves designing the interfacing hardware and firmware.
	
	\section{Hardware implementation}
	
	Below is a schematic of the designed circuit.
	
	The microcontroller is connected to the power supply through a 100 nF
	decoupling capacitor to protect it from noise. The reset pin is pulled low
	through a 10 k\textOmega {} resistor. The resistor, together with a 10
	\textmu F electrolytic capacitor, forms a power-on reset circuit. Clock
	signal is provided by a 12 MHz crystal oscillator. This results in a
	processing speed of 1 machine cycle per microsecond.
	
	The LCD is also connected to the power rail through a decoupling capacitor.
	This is to protect its internal logic circuit from noise. The anode of the
	backlight LED array is connected through a 5.6 \textOmega {} resistor,
	providing a current of about 140 mA as prescribed in the datasheet.
	A potentiometer-based voltage divider is connected to the $V_o$ pin,
	allowing the user to adjust the contrast.
	
	Finally, the 8-bit data bus of the display is connected to port 1 of the
	microcontroller. When in 4-bit mode, only 4 pins of the bus are utilized.
	The register select (RS) and chip enable pin (E) are connected to pins 1
	and 0 of the third port. The read pin is grounded -- no read operations are
	performed.
	
	\includepdf[landscape]{schematic}
	
	\section{Firmware implementation}
	
	\subsection{Overview}
	
	Communication with the HD44780 is done by exchanging data through an 8-bit
	or a 4-bit data bus, a register select pin, a read/write pin and a chip
	enable pin.
	
	The device may receive ``instruction-type" messages and ``data-type"
	messages. The former is used for controlling the display, whereas the
	latter is used to exchange text data. Every message has a predefined
	execution time which must elapse between subsequent messages.
	
	Sending a message involves appropriately setting the RS and RW pins and
	outputting the data octet through the data bus. Following this, enabling
	and disabling the E pin causes the display to read the message and commence
	execution. In 4-bit mode, the two nibbles of the octet are sent in
	sequence.
	
	\subsection{Generic code}
	
	\subsubsection{Microsecond wait subroutine (\texttt{waitUs})}
	
	This subroutine is used to introduce delay between commands with short
	execution time. It takes an argument in the A register which denotes the
	wait time in microseconds.
	
	Delay is introduced by placing a value in a register and burning machine
	cycles in a \texttt{djnz} loop. Because a single \texttt{djnz} instruction
	takes two cycles to execute, the value in the register must be divided by
	two.
	
	The preprocessing required to calculate the wait cycle count, together with
	the \texttt{call}-\texttt{ret} combo, introduce a fixed-size overhead. This
	overhead is eliminated by subtracting a constant from the argument before
	division. Thanks to this subtraction, the maximum absolute error of the
	subroutine is 1 \textmu s for odd arguments.
	
	\subsubsection{Millisecond wait subroutine (\texttt{waitMs})}
	
	This subroutine works in a similar way to \texttt{waitUs}, except that it
	uses nested loops to achieve longer delay times. It is used for the
	initialization delay and for commands with long execution time.
	
	The internal loop is fine-tuned to generate a delay of 250 ms. The argument
	of the subroutine is multiplied by two and used as the loop count of the
	external loop, which produces a delay of $A \times 2 \times 2 \times 250$
	\textmu s = $A$ ms.
	
	Because it would be difficult and of little benefit to eliminate the
	preprocessing overhead, the subroutine has a constant absolute error of
	6 \textmu s.
	
	\subsubsection{8-bit data send subroutine (\texttt{sendData})}
	
	The caller places the data in the output port register and performs the
	call. The subroutine enables the RS pin to denote a ``data-type" message
	and toggles the E pin. All data messages require a delay of 40 \textmu s,
	which is achieved by a call to \texttt{waitUs}.
	
	\subsubsection{8-bit instruction send subroutine (\texttt{sendComm})}
	
	This subroutine is analogous to \texttt{sendData}, except it drives the RS
	pin low. Following the E pin flash, the wait time is determined. If the
	instruction is a ``clear display" or a ``cursor home" command, a delay
	of 2 ms must be introduced.
	
	Three valid codes correspond to the long-wait instructions: \texttt{0x01},
	\texttt{0x02} and \texttt{0x03}. A \texttt{dec}-\texttt{jz} chain is used
	to detect these values and change the flow of execution.
	
	\subsubsection{Initialization routine}
	
	In all three (sub-)tasks, the program begins with the display
	initialization routine. This involves a delay of 30 ms and the transmission
	of four instruction messages: function set, display on, clear display,
	entry mode set. These are performed using \texttt{waitMs} and
	\texttt{sendComm}.
	
	\subsection{Task-specific code}
	
	\subsubsection{Task 1a}
	
	Because of the modular nature of the program, this task introduces very
	little purpose-built code. A single call to \texttt{sendData} is made to
	type a character on the display.
	
	\subsubsection{Task 1b}
	
	This task uses the same code as task 1a, except that the \texttt{sendData}
	and \texttt{sendComm} subroutines are modified to operate in 4-bit mode.
	Similarly, the initialization routine is prepended with code to force the
	display into 4-bit mode.
	
	The display starts in 8-bit mode by default and must be switched into 4-bit
	mode by sending the ``function set" instruction. This involves outputting
	the high nibble of the instruction to the 4-bit data bus and toggling the E
	pin. The display recognizes the transition into 4-bit mode and discards the
	lower nibble (which was read from unconnected pins DB0 through DB3).
	
	Following the transition, the initialization procedure is carried out
	according to new transmission rules. The microcontroller must toggle the E
	pin after sending each nibble of the message. The second falling edge
	begins the execution of the command.
	
	For simplicity, arguments to the \texttt{sendData} and \texttt{sendComm}
	subroutines are still written to the digital output ports. A real
	implementation could use a separate register to pass the arguments and use
	the freed 4 pins for something else.
	
	\subsubsection{Task 2}
	
	Task 2 introduces a subroutine to send character strings to the display as
	well as a call to the ``set DDRAM address" instruction to change the
	position of the cursor.
	
	The strings are stored at the beginning of the program code memory. To
	prevent the CPU from executing them as code, a jump instruction is placed
	at the very beginning of the code. The strings are defined with the
	\texttt{DB} assembler directive and zero-terminated.
	
	The main routine sets the DDRAM address to \texttt{0x4}, which corresponds
	to the fifth column of the first row. Then, the address of the first
	string is loaded into the data pointer register and a call to the newly
	introduced \texttt{sendString} subroutine is made.
	
	The \texttt{sendString} subroutine maintains an index which it uses to
	iterate over the string. Data is retrieved from memory using the
	\texttt{movc} instruction and sent to the display by means of
	\texttt{sendData}. If the retrieved byte is zero, the subroutine returns;
	otherwise, the index is incremented and the loop reiterates.
	
	Finally, the cursor is moved to the beginning of the second row (address
	\texttt{0x40}) and the second string is sent.
	
	\section{Final questions}
	
	\subsubsection*{How is 4-bit mode configured in the LCD?}
	
	The 4-bit mode is configured by sending the ``function set" instruction to
	the display. The high nibble contains information on whether the 8-bit or
	the 4-bit mode is used. The display discards the low nibble and switches
	to 4-bit mode.
	
	\subsubsection*{Why is it convenient to use a 12 MHz system clock with
	8051 microcontrollers?}
	
	On the 8051, a single machine cycle corresponds to 12 clock cycles. At 12
	MHz, this means that a single instruction is executed in either 1 or 2
	\textmu s. This is useful for timing.
	
	\subsubsection*{Are delay methods presented in the tutorial part optimal?
	Is it possible to generate precise time intervals using loops and
	\texttt{nop} commands?}
	
	They are optimal in that they use nested loops, which saves memory and
	gives long delays. It's possible to fine-tune a wait subroutine to have a
	maximum error of 1 \textmu s, as demonstrated in this report.
	
	\subsubsection*{How many $2 \times 16$ displays can you connect to the
	AT89C4051 microcontroller without using any additional ICs?}
	
	You can connect up to 10 displays. The AT89C4051 has 15 general-purpose IO
	pins. You can use 5 of them for a common DB/RS bus and the remaining 10 to
	enable writing to individual displays.
	
	\section{Declaration of authorship}
	
	I declare that this piece of work (solving laboratory 1) which is the basis
	for recognition of achieving learning outcomes in the Microprocessor
	Systems (EMISY) course was completed on my own.
	
	\noindent
	Michał Szopiński 300182
	
\end{document}
