\documentclass{article}

\usepackage{graphicx}
\usepackage{pdfpages}
\usepackage{hyperref}
\usepackage{textgreek}
\setlength{\parskip}{1em}

\begin{document}

	\title{EMISY lab 3 report:\\
	Interfacing the 8051 MCU with a keypad and a switch}
	\author{Michał Szopiński 300182\\\\
	https://github.com/Lachcim/szopinski-emisy-lab}
	\date{May 24, 2021}
	\maketitle
	
	\setcounter{section}{-1}
	\section{Abstract}
	
	The goal of this laboratory was to connect the AT89C4051 (Intel 8051
	family) microcontroller to a keypad and a switch and to observe the changes
	in the input.
	
	\section{Hardware implementation}
	
	A switch was connected to one of the pins of port 2. When open, the pin is
	pulled high through a pull-up resistor. When closed, the pin is pulled low
	through the switch.
	
	The LCD module was connected in the 8-bit configuration. Port 1 serves as
	the data bus and two pins from port 3 control the RS and E inputs.
	
	The keypad was connected to seven pins of port 0. This configuration would
	allow the keys to be scanned and read individually; this, however, was not
	necessary in this laboratory. The three outputs of the keypad are connected
	to an AND gate to trigger an external interrupt each time a key is pressed.
	
	Finally, an LED was connected to the MCU with the cathode facing the output
	pin.
	
	\section{Firmware implementation}
	
	\subsection{Task 1}
	
	In this task, the LCD display is to display a message each time a switch is
	toggled on, and to clear the screen each time the switch is disabled.
	
	This is done by simply waiting for the switch to come on, calling the
	string displaying subroutine from lab 1, waiting for the switch to turn
	off, calling the screen clearing subroutine, and looping over.
	
	\subsection{Task 2}
	
	In this task, an LED is to be toggled each time a key is pressed.
	
	All of the row pins of the keypad are driven low, which causes the column
	pins to be pulled low whenever a key is pressed. External interrupt 1 is
	configured to trigger on the falling edge of the input signal. As a result,
	the interrupt fires on each key press.
	
	In the interrupt, a logical XOR is performed on the LED port, which causes
	the LED to toggle.
	
	\section{Declaration of authorship}
	
	I declare that this piece of work (solving laboratory 1) which is the basis
	for recognition of achieving learning outcomes in the Microprocessor
	Systems (EMISY) course was completed on my own.
	
	\noindent
	Michał Szopiński 300182
	
\end{document}
