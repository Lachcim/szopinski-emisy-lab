\documentclass{article}

\usepackage{graphicx}
\usepackage{pdfpages}
\usepackage{hyperref}
\usepackage{textgreek}
\setlength{\parskip}{1em}

\begin{document}

	\title{EMISY lab 2 report:\\
	Interfacing the 8051 MCU with a 7-segment display}
	\author{Michał Szopiński 300182\\\\
	https://github.com/Lachcim/szopinski-emisy-lab}
	\date{May 10, 2021\\}
	\maketitle
	
	\setcounter{section}{-1}
	\section{Abstract}
	
	The goal of this laboratory was to connect the AT89C4051 (Intel 8051
	family) microcontroller to a generic quadruple 7-segment display. This
	involves designing the interfacing hardware and firmware.
	
	\section{Hardware implementation}
	
	Below is a schematic of the designed circuit. It is common for all tasks.
	
	The seven-segment display uses the ``common anode" configuration, which
	enables the microcontroller to sink currents of up to 10 mA. The cathode
	bus is used for enabling individual segments in a single display. A decoder
	enables current to each of the displays on demand through a PNP transistor.
	The currently enabled display is selected by sending its address through
	a 2-bit address bus and raising the chip select pin of the decoder.
	
	\includepdf[landscape]{schematic}
	
	\section{Firmware implementation}
	
	\subsection{Task 1}
	
	The goal of this task was to design a subroutine whose purpose is to
	display the specified pattern of segments on the given display. The
	subroutine accepts two arguments describing the bit pattern as an 8-bit
	value and an address as a 2-bit integer ranging from 0 to 3.
	
	It starts by disabling the chip select pin of the decoder, blanking all
	displays. Then, the supplied bit pattern is inverted, marking enabled
	segments with zeros and disabled segments with ones. The inverted value is
	moved to the segment bus output port.
	
	To simplify addressing, the lowest 2 bits of port 1 were chosen as the
	address bus for the decoder. The two-bit address may be easily outputted
	through port 1 by performing bitwise operations on its SFR.
	
	Once the bit pattern and the display address are ready, the decoder is
	reenabled and the pattern appears on the display.
	
	\subsection{Task 2}
	
	The goal of this task was to blink a LED on one of the displays in
	intervals of 20 ms. The code starts with a configuration procedure in which
	timer 0 of the 8051 is configured as a 16-bit non-gated timer. Global
	interrupts are enabled and the timer overflow interrupt is enabled as well.
	
	The overflow interrupt occurs when the timer register exceeds $2^{16} - 1$
	and rolls over. To ensure intervals of 20 ms, the initial value of the
	timer is set to $2^{16} - 20000 + 7$, with the value $7$ to account for
	overhead from executing the handler.
	
	At the end of the initial procedure, the decoder is enabled and the
	leftmost display is selected. The microcontroller then enters an infinite
	no-op loop and lets interrupt handlers do the rest.
	
	The handler toggles one of the pins of the segment bus. Before it exits, it
	resets the timer to the previously calculated value.
	
	As a side note, the suggestion to ``clear overflow flag" as stated in the
	manual appears to be pointless since the flag is automatically cleared
	when the interrupt handler is vectored.
	
	\subsection{Task 3}
	
	The goal of this task was to display the final four digits of the student's
	album number on the 7-segment display. Because EDSIM doesn't support visual
	interlacing of individual displays, the digits must be displayed
	sequentially in noticeable intervals.
	
	The main functionality was implemented in a similar way to the one
	presented in task 1. A dedicated subroutine displays the specified digit at
	the specified position. It uses a lookup table to convert a decimal value
	to the right segment pattern.
	
	The code starts by configuring the timer in the same way as it was done in
	task 2, although this time letting it overflow after the natural boundary
	of $2^{16} - 1$ cycles. This yields an interrupt rate of once per 65 ms,
	observable in EDSIM.
	
	The interrupt handler iterates over the album number digits stored in
	program memory. The iterator is used for retrieving the digits and
	determining their position on the display. Both of these parameters are
	passed as arguments to the digit displaying subroutine. At the end of the
	handler, the iterator is decremented and its modulo 4 is taken.
	
	\section{Final questions}
	
	\subsubsection*{Describe how is a display selected in EDSIM simulator?}
	
	EDSIM uses a decoder chip to convert a 2-bit address to one of the 4
	active-low outputs, each of which enables current to a single display. The
	chip select pin enables the decoder.
	
	\subsubsection*{What is the difference between common anode and common
	cathode LED displays in terms of interfacing them with MCU?}
	
	Common anode displays must have their cathodes switched to ground to
	enable them, whereas common cathode displays require that their anodes are
	switched to the power rail. The common anode configuration allows for a
	greater current flowing through the MCU (for 8051 chips).
	
	\subsubsection*{How can be a timer peripheral used to generate precise time
	intervals (precise delays)?}
	
	The timer counter register may be set to a pre-calculated value each time
	the overflow interrupt handler fires, which allows the rate of overflow
	interrupts to be accurately controlled.
	
	\subsubsection*{How does multiplexing driving mode of LED displays work?
	Compare it with static driving mode of LED displays in terms of required
	GPIO pins and current consumption. When does it make sense to use
	multiplexing and when does it not?}
	
	LED displays are multiplexed by rapidly switching between individual
	digits, so that the human eye cannot perceive any blinking. Static driving
	requires more pins and draws more current due to a higher number of enabled
	LEDs. Multiplexing is generally the preferred method, albeit static mode
	may be required for when the display is viewed through a camera.
	
	\section{Declaration of authorship}
	
	I declare that this piece of work (solving laboratory 1) which is the basis
	for recognition of achieving learning outcomes in the Microprocessor
	Systems (EMISY) course was completed on my own.
	
	\noindent
	Michał Szopiński 300182
	
\end{document}
